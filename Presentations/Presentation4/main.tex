\documentclass{beamer}
\mode<presentation>
\usepackage{amsmath}
\usepackage{amssymb}
\usepackage{mathtools}
\usepackage{listings}
\usepackage{graphicx}
\usepackage{hyperref}
\usetheme{Boadilla}
\usecolortheme{lily}

\setbeamertemplate{footline}{
  \leavevmode%
  \hbox{% 
    \begin{beamercolorbox}[wd=.9\paperwidth,ht=2.25ex,dp=1ex,left]{author in head/foot}%
      \hspace{1em} Arnav Makarand Yadnopavit
    \end{beamercolorbox}%
    \begin{beamercolorbox}[wd=.1\paperwidth,ht=2.25ex,dp=1ex,right]{author in head/foot}%
      \insertframenumber{} / \inserttotalframenumber\hspace*{2ex}
    \end{beamercolorbox}}%
}
\setbeamertemplate{navigation symbols}{}

\numberwithin{equation}{section}
\title{10.3.2.5: LU Factorization}
\author{EE24BTECH11007 - Arnav Makarand Yadnopavit}
\date{\today}

\begin{document}

\begin{frame}
\titlepage
\end{frame}

\section*{Outline}
\begin{frame}
\tableofcontents
\end{frame}

\section{Question}
\begin{frame}
\frametitle{Question}
Half the perimeter of a rectangular garden, whose length is 4 m more than its width, is 36 m. Find the dimensions of the garden.
\end{frame}

\section{Solution}
\subsection{Matrix Representation}
\begin{frame}
\frametitle{Matrix Representation}
Let the length and width of the garden be $x$ and $y$, respectively.
\begin{align*}
    x + y &= 36, \\
    x - y &= 4.
\end{align*}
We represent this system in matrix form:
\begin{align*}
    A &= \begin{bmatrix} 1 & 1 \\ 1 & -1 \end{bmatrix}, \quad
    b = \begin{bmatrix} 36 \\ 4 \end{bmatrix}, \quad
    x = \begin{bmatrix} x \\ y \end{bmatrix}.
\end{align*}
\end{frame}

\subsection{LU Factorization}
\begin{frame}
\frametitle{LU Factorization Using Update Equations}
\begin{itemize}
    \item Given a matrix $\mathbf{A}$ of size $n \times n$, LU decomposition is performed row by row and column by column.
    \item The update equations are as follows:
\end{itemize}
\end{frame}

\begin{frame}
\frametitle{Step-by-Step Procedure}
\begin{enumerate}
    \item \textbf{Initialization:} 
    \begin{itemize}
        \item Start by initializing $\mathbf{L}$ as the identity matrix $\mathbf{L} = \mathbf{I}$ and $\mathbf{U}$ as a copy of $\mathbf{A}$.
    \end{itemize}
    \item \textbf{Iterative Update:}
    \begin{itemize}
        \item For each pivot $k = 1, 2, \ldots, n$:
        \begin{enumerate}
            \item Compute the entries of $\mathbf{U}$ using the first update equation.
            \item Compute the entries of $\mathbf{L}$ using the second update equation.
        \end{enumerate}
    \end{itemize}
    \item \textbf{Result:}
    \begin{itemize}
        \item After completing the iterations, the matrix $\mathbf{A}$ is decomposed into $\mathbf{L} \cdot \mathbf{U}$, where $\mathbf{L}$ is a lower triangular matrix with ones on the diagonal, and $\mathbf{U}$ is an upper triangular matrix.
    \end{itemize}
\end{enumerate}
\end{frame}

\subsection{Update Equations}
\begin{frame}
\frametitle{Update for $U_{k,j}$ (Entries of $U$)}
For each column $j \geq k$, the entries of $U$ in the $k$-th row are updated as:
\[
    U_{k,j} = A_{k,j} - \sum_{m=1}^{k-1} L_{k,m} \cdot U_{m,j}, \quad \text{for } j \geq k.
\]
This equation computes the elements of the upper triangular matrix $\mathbf{U}$ by eliminating the lower triangular portion of the matrix.
\end{frame}

\begin{frame}
\frametitle{Update for $L_{i,k}$ (Entries of $L$)}
For each row $i > k$, the entries of $L$ in the $k$-th column are updated as:
\[
    L_{i,k} = \frac{1}{U_{k,k}} \left( A_{i,k} - \sum_{m=1}^{k-1} L_{i,m} \cdot U_{m,k} \right), \quad \text{for } i > k.
\]
This equation computes the elements of the lower triangular matrix $\mathbf{L}$, where each entry in the column is determined by the values in the rows above it.
\end{frame}

\subsection{LU Decomposition Result}
\begin{frame}
\frametitle{LU Decomposition Result}
Using code, we compute:
\[
    L = \begin{bmatrix} 1 & 0 \\ 1 & 1 \end{bmatrix}, \quad
    U = \begin{bmatrix} 1 & 1 \\ 0 & -2 \end{bmatrix}.
\]
\end{frame}

\subsection{Solving $Ax = b$}
\begin{frame}
\frametitle{Forward Substitution: Solve $Ly = b$}
\begin{align*}
    \begin{bmatrix} 1 & 0 \\ 1 & 1 \end{bmatrix} \begin{bmatrix} y_1 \\ y_2 \end{bmatrix} = \begin{bmatrix} 36 \\ 4 \end{bmatrix}.
\end{align*}
\begin{itemize}
    \item From the first row: $y_1 = 36$.
    \item From the second row: $y_1 + y_2 = 4 \implies y_2 = -32$.
\end{itemize}
Thus:
\[
    y = \begin{bmatrix} 36 \\ -32 \end{bmatrix}.
\]
\end{frame}

\begin{frame}
\frametitle{Back Substitution: Solve $Ux = y$}
\begin{align*}
    \begin{bmatrix} 1 & 1 \\ 0 & -2 \end{bmatrix} \begin{bmatrix} x \\ y \end{bmatrix} = \begin{bmatrix} 36 \\ -32 \end{bmatrix}.
\end{align*}
\begin{itemize}
    \item From the second row: $-2y = -32 \implies y = 16$.
    \item Substitute $y = 16$ into the first row: $x + y = 36 \implies x = 20$.
\end{itemize}
Thus:
\[
    x = 20, \quad y = 16.
\]
\end{frame}

\section{Graphical Representation}
\begin{frame}
\frametitle{Graphical Representation}
\begin{figure}[h]
    \centering
    \includegraphics[width=0.8\linewidth]{figs/fig.png}
\end{figure}
\end{frame}

\end{document}

